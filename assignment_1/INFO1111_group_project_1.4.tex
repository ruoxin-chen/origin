\documentclass[a4paper, 11pt]{report}
\usepackage{blindtext}
\usepackage[T1]{fontenc}
\usepackage[utf8]{inputenc}
\usepackage{titlesec}
\usepackage{fancyhdr}
\usepackage{geometry}

\usepackage[english]{babel}
\usepackage{apacite}

\geometry{ margin=30mm }
\counterwithin{subsection}{section}
\renewcommand\thesection{\arabic{section}.}
\renewcommand\thesubsection{\thesection\arabic{subsection}.}
\usepackage{tocloft}
\renewcommand{\cftchapleader}{\cftdotfill{\cftdotsep}}
\renewcommand{\cftsecleader}{\cftdotfill{\cftdotsep}}
\setlength{\cftsecindent}{2.2em}
\setlength{\cftsubsecindent}{4.2em}
\setlength{\cftsecnumwidth}{2em}
\setlength{\cftsubsecnumwidth}{2.5em}


\begin{document}
\titleformat{\section}
{\normalfont\fontsize{15}{0}\bfseries}{\thesection}{1em}{}
\titlespacing{\section}{0cm}{0.5cm}{0.15cm}
\titleformat{\subsection}
{\normalfont\fontsize{13}{0}\bfseries}{\thesubsection}{0.5em}{}
\titlespacing{\section}{0cm}{0.5cm}{0.15cm}

%=======================================================================================

\begin{titlepage}
\center 
\textbf{\huge INFO1111: Computing 1A Professionalism}\\[0.75cm]
\textbf{\huge 2022 Semester 1}\\[2cm]
\textbf{\huge Practice: Team Project Report}\\[3cm]

\textbf{\huge Submission number: 1}\\[0.75cm]
\textbf{\huge Team Members:}\\[0.75cm]
\textbf{\large
    \begin{tabular}{|p{0.5\textwidth}|p{0.3\textwidth}|p{0.2\textwidth}|}
        \hline
        Name & Student ID & Levels being attempted in this submission\\
        \hline
        ?? & ?? & e.g. 1,2 \\
        Felicia Chen & 520400139 & 1 \\
        ?? & ?? & ?? \\
        ?? & ?? & ?? \\
        \hline
    \end{tabular}
}\\[0.75cm]
\end{titlepage}

%=======================================================================================

\tableofcontents

%=======================================================================================

\newpage
\section*{General Instructions}

You should use this \LaTeX\ template to generate your team project report. Keep in mind the following key points:
\begin{itemize}
    \item When we assess your report, you are not given a mark. Instead we will indicate (separately, for each team member) whether each level is ''achieved''.
    \item In order to pass the unit, you must achieve at least level 1. 
    \item In order to achieve level 2, you must first have achieved level 1, and so on for each level up to level 4. This means that we will not assess a higher level until a lower level has been achieved (though we will review one level higher and give you feedback to help you in refining your work).
    \item Some parts of the report are completed as a team and other parts require each student to complete a different section. This means that for each submission, some members of the team may have completed their work for a given section, but other members may not. It also is therefore possible that some members of the team may achieve a specified level and other members of the team may not yet have achieved that level.
    \item Even if some members are completing their material for a given level, and others are not, your team members will still need to work together to edit and compile the report.  The only exception to this is where a member of the team has already achieved the level they are targeting in a previous submission and has decided to not attempt higher levels, and so is not contributing any further (this should be obvious because no level is indicated for that student on the cover page).
    \item When completing each section you should remove the explanation text and replace it with your material.
\end{itemize}

For each submission you will add new details to this report, and/or update previous sections (where previous work was not good enough to have achieved the relevant level). In particular:

\begin{itemize}
    \item \textbf{General:} For each submission, each student can attempt up to 2 levels. You must also successfully achieve each lower level before you can be assessed at a higher level. For example, in the first submission you might attempt only level 1, but not be successful in achieving that level. You then reattempt level 1 and add in level 2 in the second submission and are successful in achieving level 1 but not level 2. For the third and final submission you could then attempt level 2, or levels 2 and 3 - or even just choose to not submit anything further and remain at level 1).
    \item \textbf{Submission 1:} You should complete at least the material for level 1 (since achieving level 1 is required to pass the unit). Each member of the team can also optionally choose to complete the material for level 2.\\
    \textit{Note 1: If you do not complete the level 2 information then you obviously cannot achieve level 2 at this stage. This does not stop you from attempting level 2 in Deliverable 2 or 3, but it will make it more difficult to achieve the higher levels later in the semester.}
    \textit{Note 2: To be able to achieve Level 1 in submission one your team has to achieve level 1 in the group component (Section 1.1) and you have to achieve Level 1 in the individual component (i.e. your assigned section 1.2, 1.3, 1.4 or 1.5)}
    \item \textbf{Submission 2:} Each member of your team will complete additional sections, but because you are submitting a single document, you need to work together to compile your results together and generate the final submission.\\
    If you did not achieve level 1 in your first submission, then you should revise the material for level 1 based on the feedback, and optionally you can also complete level 2.\\
    If you achieved level 1 in your first submission, then each team member can optionally complete the material for levels 2 and 3.
    \textit{Note: If you do not achieve level 1 with this submission then the highest level you will be able to achieve in the final submission will be level 2. If you achieve level 1, but not level 2, with this submission then the highest level you will be able to achieve with the final submission is level 3.}
    \item \textbf{Submission 3:} Again, you can correct sections where you did not achieve the specified level in the previous submission, and you complete additional sections.\\
    If you still have not achieved level 1, then you should revise the material for level 1 based on the feedback, and again optionally you can also complete level 2.\\
    For those at level 1, you can choose to complete the material for levels 2 and 3.\\
    For those at level 2, you can choose to complete the material for levels 3 and 4.\\
    For those at level 3, you can choose to complete the material for level 4.
\end{itemize}

Whilst the team project is just that -- a team project -- it has been designed to also allow different members of the team to achieve different outcomes. We do expect you to work together as a team. If you do come across problems working together then the first step should be to discuss this with your tutor. Note: If you are having problems you should approach your tutor as soon as you can to make them aware of the difficulties you are having with your team.

Finally, you should also ensure that any resources you use are suitably referenced, and references are included into the reference list at the end of this document. You should use APA 6th reference style \cite{apa6}.

%=======================================================================================

\newpage
\section{Level 1: Basic Skills}

Level 1 focuses on basic technical skills (related to \LaTeX\ and Git) and the types of skills used in different computing jobs.

\subsection{Developing industry skills}

This section is completed as a team.\\
Throughout your Computing degree we will help you learn a range of new skills. Once you graduate however you will need to continue to learn new languages, new tools, new applications, etc. For this section you need to identify 5 approaches you can take to this continual learning. You should then put these in order from most effective to least effective, and then explain the circumstances in which each approach might be appropriate. (Target = $\sim$100 words per skill = $\sim$500 words total).

\subsection*{Online Courses}
Taking paid or free online courses to learn new skills is a very effective strategy, particularly when we try to learn a completely new topic and may not even know where to start. MOOC (Massive open online courses) platforms, such as Coursera, EdX, Udemy, and Code Academy \cite {Chaudhry_2020}, provide free courses that can guide us through new topics. These courses usually include sets of lecture videos, interactive quizzes, auto-graded exercises, links for further studies and online discussion platforms which help us obtain new skills. Further, since it is self-paced learning, it offers great flexibility, enabling people to balance work and study. 
\subsection*{Official Documentation}
Referring to official documentation is effective when we troubleshoot the problems encountered when using a new tool. Official documentation, which are essentially user guides, explain and illustrate how a product works. For instance, the documentation for Python 3.10.4 includes  a Python tutorial which illustrates key features and concepts with many code examples, a summary of new features of the new version, library reference, usage, setup, FAQs, etc \cite {PythonSoftwareFoundation_2019}. All these resources provide guidance on how to use the product correctly. However, documentation usually includes jargon that new users do not know. Therefore, users will need to fill in these knowledge gaps before following the documentation. 

\subsection{Skills: add student 1 name here : Computer Science}

This section is completed individually. Each member of the team should independently complete a separate copy of this section.\\
You should begin by allocating to each team member a different major to focus on (i.e. one of: Computer Science; Data Science; Software Development; Cyber Security). \textit{If you have a fifth member, then your tutor will suggest a fifth topic to cover}. You should then undertake research into the typical practical skills that you believe would be most important to someone who graduates with this major and is then working in industry. You should list the 8 skills that you believe are most important and for each one give a short explanation as to why you feel it is important. (Target = $\sim$100 words per skill $\sim$800 words total per student).

\subsection{Skills: add student 2 name here : Data Science}

Your text goes here

\subsection{Skills: Felicia Chen : Software Development}
\subsubsection{Communication Skill}
Communication skill is important because the first stage of software development involves identifying software requirements. Software requirement specification includes the product’s purpose, functionality, features, intended users, operating environment, interfaces, etc. For example, to design an inventory management software for businesses, it is crucial to communicate with clients to figure out their requirements which, in this case, may be a product tracking function to manage stocks. Sometimes the clients could have vague ideas of desired end results or even contradictory requirements. Then software developers need to interact with them for clarifications so as to develop the product that meets customised requirements. 
\subsubsection{Teamwork Skill}
Developing a software, particularly a large project, is typically done by a team of people who are responsible for different parts of software development. Generally, a team consists of a project manager, a software architect who creates solutions to meet requirements and defines the overall architecture, developers who write codes, testers who run test cases and identify defects or bugs, quality assurance team, UI and UX designers who analyse its functionality and design the user interface, and scrum masters who act as team facilitators \cite {Paschetta_2021}. To successfully create and maintain a software, team members need to collaborate, compromise, and motivate each other throughout the process.
\subsubsection{Programming Languages}
Since software development involves writing, debugging and maintaining the source code, familiarising with various programming languages and at least mastering one is an essential skill. The choice of the programming language depends on the type of software developed. For instance, for system softwares, C++ is most often used, whereas for application softwares, Java, Python and C++ are common choices. Apart from specialising in a language, writing clear, readable and extendable code is also important, particularly when the program needs to be modified and maintained. Moreover,  a software developer should be able to identify and fix bugs because errors are inevitable in the development process. 
\subsubsection{Software development tools}
Working in the software industry requires the technical skill of using a range of tools. For instance, GitHub which hosts Git repositories is widely used because it facilitates version control and collaboration between developers. Text editors boost the efficiency of writing codes, while IDEs such as Eclipse provide code editor, automation tools, compiler, and debugger \cite {white_2021}. SQL, a database programming language to manage and retrieve data, is used in many projects, particularly in data-driven applications. Cloud platforms such as AWS are useful since many companies rely on cloud services. Therefore, being able to apply these tools and choose which ones to use depending on the task is crucial. 
\subsubsection{Data structure and Algorithm}
Software developers must be equipped with comprehensive knowledge of data structure and algorithm, which is the foundation for writing codes, before developing applications. This is because mastering data structure and algorithms allows developers to optimise their codes, and thus, have a better solution to the problem. Without knowing how algorithms can be used to perform operations on the stored data, the code written can take too much time to run or take up too much memory, especially when dealing with data \cite {upadhyay_2019}. Further, a scalable algorithm is particularly desirable when there is a large set of data inputs, which means it can solve problems of larger size. 
\subsubsection{Software artefacts}
Since software artefacts are important reference material for developers throughout the development and maintenance process, being able to create software artefact is a valuable skill. Artefacts can take various forms, including setup scripts, diagrams, prototypes, design documents, etc \cite {gillis_2022}. Artefacts describe the software’s architecture and functionality, thus serving as a guide or template for developing a program. Further, artefacts improve the continuity and maintainability of software since developers can refer to them when updating programs or correcting bugs that arise later. Program prototypes demonstrate the product’s functionality to clients, thus allowing for feedback in an early stage so that developers could modify it according to their requirements. 
\subsubsection{Evaluation Skill}
Software evaluation is an important part of its development and maintenance. Through evaluation, developers can determine whether the product fulfils the client’s requirements and make decisions if it needs any improvement. There are many criterias to assess a software, including usability, sustainability, and maintainability \cite {jackson_2011}. For example, whether the software can run on different operating systems and browser applications (portability),  whether making any modifications on its functionality is straightforward (changeability), how enjoyable the customer experience is and how to improve it, whether it is efficient and productive, etc. Therefore, software developers should learn to evaluate software since evaluation enables continuous enhancement being made. 
\subsubsection{Self-learning Skill}
The technological landscape for software development is constantly evolving. To keep up with emerging technology, self-learning skill is indispensable for software developers who need to self-teach and assimilate new technology quickly. For example, as we rely more on the cloud, cloud-based CI (continuous integration) / CD (continuous delivery) tools and multi-cloud architectures which involve using multiple cloud computing services have been gaining popularity. Meanwhile, dominating programming languages such as Java and Python may be replaced by new languages such as Go, Ruby, and Kotlin in the near future \cite{wickramasinghe_2021}. Since prominent tools and practices could become obsolete soon, software developers should have strong self-learning skills. 

\subsection{Skills: add student 4 name here : Cyber Security}

Your text goes here


%=======================================================================================

\newpage
\section{Level 2: Basic Technology}

Level 2 focuses on initial evaluation of the tech stack that is used by a selected company. All companies make use of a range of technologies, and these technologies need to work together. A tech stack is basically just this collection of technologies that collectively enable a company's systems. As an example, one of the most common technology stacks for supporting web servers is LAMP: Linux as the underlying operating system; Apache as a web server; MySQL as the supporting database; and Perl (or more recently PHP or Python) as the programming language.

Each student should choose a different tech stack and explain the role of each of the different technologies in that stack. Note that prior to researching your proposed tech stack and spending time writing about it, it might be a good idea to check with your tutor as to whether your chosen stack is suitable. (Target = $\sim$200-400 words per student).

\subsection{Tech Stack: add student 1 name here}

Your text goes here

\subsection{Tech Stack: add student 2 name here}

Your text goes here

\subsection{Tech Stack: add student 3 name here}

Your text goes here

\subsection{Tech Stack: add student 4 name here}

Your text goes here


%=======================================================================================

\newpage
\section{Level 3: Advanced Skills}

Level 3 focuses on more advanced technical skills (\LaTeX\ and Git) and analysis of linkages and relationships between the items in the company tech stack.

The following is a list of advanced Git and \LaTeX\ skills/features. Each student should select one pair of items from each list and demonstrate actual use of each item (either through activity in Git, or through including items in this report). (Target = $\sim$100 words per student for each feature).
\begin{itemize}
    \item Git
    \begin{itemize}
        \item Rebasing and Ignoring files
        \item Forking and Special files
        \item Resetting and Tags
        \item Reverting and Automated merges
        \item Hooks and Tags
    \end{itemize}
    \item \LaTeX\ 
    \begin{itemize}
        \item Cross-referencing and Custom commands
        \item Footnotes/margin notes and creating new environments
        \item Floating figures and editing style sheets
        \item Graphics and advanced mathematical equations
        \item Macros and hyperlinks
    \end{itemize}
\end{itemize}

\subsection{Advanced features: add student 1 name here}

Explain your use of the advanced Git and \LaTeX\ features. 

\subsection{Advanced features: add student 2 name here}

Explain your use of the advanced Git and \LaTeX\ features. 

\subsection{Advanced features: add student 3 name here}

Explain your use of the advanced Git and \LaTeX\ features. 

\subsection{Advanced features: add student 4 name here}

Explain your use of the advanced Git and \LaTeX\ features. 



%=======================================================================================

\newpage
\section{Level 4: Advanced Knowledge}

Level 4 focuses on analysing your particular tech stack and considering alternatives. Each student should consider the tech stack they described for Level 2, and then discuss each of the following points:
\begin{itemize}
    \item What are the strengths and limitations of this stack? (Target = $\sim$200 words).
    \item What alternatives exist, and under what situations might these alternatives be a better choice? (Target = $\sim$200 words).
\end{itemize}

\subsection{Advanced Knowledge: add student 1 name here}

Your text goes here

\subsection{Advanced Knowledge: add student 2 name here}

Your text goes here

\subsection{Advanced Knowledge: add student 3 name here}

Your text goes here

\subsection{Advanced Knowledge: add student 4 name here}

Your text goes here



%=======================================================================================

\newpage

\bibliographystyle{apacite}
\bibliography{main}

\end{document}
\end{report}
